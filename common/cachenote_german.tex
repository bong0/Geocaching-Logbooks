GEOCACHE Versteck - Bitte lesen

Herzlichen Glückwunsch, Du hast es gefunden! Beabsichtigt oder nicht!

Warum liegt hier dieser Behälter? Was zum Teufel macht dieser Behälter hier mit den ganzen komischen Dingen drin?

Dieser Behälter ist Teil eines weltweiten Spieles, das sich der Nutzung des GPS (Global Positioning System) verschrieben hat. Es nennt sich GPS Stash Hunting, oder Geocaching. Das Spiel besteht darin, einen "Schatz" zu verstecken und dessen GPS Koordinaten zu veröffentlichen, so dass andere GPS-Benutzer diesen im Rahmen einer "Schatzsuche" finden können. 
Die einzigen Regeln sind: 
Du solltest Deinen Besuch vorort in das Logbuch eintragen. (Woher kommst Du, war es einfach, den Schatz zu finden etc.)
Wenn Du etwas aus dem Schatz entnimmst, musst Du auch etwas hinterlassen.
Hoffentlich hat die Person, die den Schatz versteckt hat, einen guten Platz in öffentlichem Gelände gefunden, um ihn zu verstecken. 
Manchmal wird jedoch aus einem guten Platz ein schlechter :-(

WENN DU DIES DURCH ZUFALL GEFUNDEN HAST

Grossartig! Du bist herzlich eingeladen mitzumachen. Wir bitten Dich nur um folgendes:
\begin{itemize*}
	\item Bitte verstecke den Behälter nicht woanders und beschädige ihn nicht. Der wahre Wert liegt im Finden des Behälters und im Austausch der Gedanken mit allen Anderen, die diesen Behälter noch finden werden.
	\item Bitte trage Deinen Besuch vorort im Logbuch ein. Wenn Du willst, entnimm etwas aus dem "Schatz". Hinterlasse aber dann bitte auch etwas im Behälter für andere Schatzsucher.
	\item Wenn möglich, lasse uns wissen, dass du den Behälter gefunden hast, indem Du Dich über die unten aufgeführte Web-Seite meldest.
\end{itemize*}
Der GPS Stash Hunt ist offen für jeden mit einem GPS und etwas Sinn für Abenteuer. Es gibt viele ähnliche Verstecke auf der ganzen Welt. Im Augenblick hat die Organisation ihren Sitz im Internet. Besuche unsere Website, wenn Du mehr wissen willst oder Kommentare hast:

\url{http://www.geocaching.com}

Wenn dieser Behälter auf privatem Grund und Boden liegt und Du möchtest, dass er entfernt wird, teile uns dies bitte mit (contact@geocaching.com). 
Wir entschuldigen uns schon im Voraus dafür.


